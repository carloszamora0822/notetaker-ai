\documentclass[11pt,letterpaper]{article}

% Modern packages
\usepackage[utf8]{inputenc}
\usepackage[T1]{fontenc}
\usepackage{lmodern}  % Better font
\usepackage{microtype}  % Better typography
\usepackage{xcolor}
\usepackage{geometry}
\usepackage{fancyhdr}
\usepackage{hyperref}
\usepackage{enumitem}  % Better lists
\usepackage{tcolorbox}  % Colorful boxes
\usepackage{graphicx}
\usepackage{amsmath,amssymb}  % Math support

% Page geometry with better margins
\geometry{
    margin=0.9in,
    top=1.1in,
    headheight=14pt
}

% Theme colors (will be replaced by generator)
\definecolor{primarycolor}{HTML}{dc3545}
\definecolor{secondarycolor}{HTML}{ff6b6b}
\definecolor{accentcolor}{HTML}{1E88E5}
\definecolor{backgroundcolor}{HTML}{F8F9FA}

% Modern header styling with gradient effect
\pagestyle{fancy}
\fancyhf{}
\fancyhead[L]{\textcolor{primarycolor}{\small\textbf{EXPERIENCING MUSIC}}}
\fancyhead[R]{\textcolor{secondarycolor!80}{\small 2025-09-30}}
\fancyfoot[C]{\textcolor{primarycolor!60}{\small\thepage}}
\renewcommand{\headrulewidth}{0.5pt}
\renewcommand{\headrule}{\hbox to\headwidth{\color{primarycolor!30}\leaders\hrule height \headrulewidth\hfill}}

% Hyperref setup
\hypersetup{
    colorlinks=true,
    linkcolor=primarycolor,
    urlcolor=secondarycolor,
    citecolor=primarycolor,
    pdfborder={0 0 0}
}

% Modern section formatting with boxes
\usepackage{titlesec}
\titleformat{\section}
  {\Large\bfseries\color{primarycolor}}
  {}{0pt}{}
  [\vspace{-8pt}{\color{primarycolor}\titlerule[1.5pt]}]

\titleformat{\subsection}
  {\large\bfseries\color{secondarycolor}}
  {\textcolor{secondarycolor}{\textbullet}\hspace{0.3em}}{0pt}{}

\titleformat{\subsubsection}
  {\normalsize\bfseries\color{accentcolor}}
  {\textcolor{accentcolor}{\tiny\blacksquare}\hspace{0.3em}}{0pt}{}

% Title styling
\title{\textcolor{primarycolor}{\Huge\textbf{EXPERIENCING MUSIC Notes}}}
\author{}
\date{\textcolor{secondarycolor}{2025-09-30}}

% Customize list styling
\setlist[itemize]{leftmargin=*, itemsep=3pt, topsep=6pt}
\setlist[enumerate]{leftmargin=*, itemsep=3pt, topsep=6pt}

% Custom box for highlights
\newtcolorbox{highlightbox}{%
    colback=secondarycolor!5,
    colframe=primarycolor,
    boxrule=0.5pt,
    arc=3pt,
    left=8pt,
    right=8pt,
    top=6pt,
    bottom=6pt
}

\begin{document}

% Modern title page with gradient background effect
\begin{tcolorbox}[
    enhanced,
    colback=backgroundcolor,
    colframe=primarycolor,
    boxrule=2pt,
    arc=8pt,
    width=\textwidth,
    left=0pt,
    right=0pt,
    top=0pt,
    bottom=0pt
]
    \begin{center}
        \vspace{1.2cm}
        % Main title with modern styling
        {\fontsize{32}{38}\selectfont\textcolor{primarycolor}{\textbf{EXPERIENCING MUSIC}}} \\[0.8cm]

        % Decorative line
        {\color{secondarycolor}\rule{0.6\textwidth}{1.5pt}} \\[0.5cm]

        % Subtitle
        {\Large\textcolor{secondarycolor!80}{\textit{Lecture Notes}}} \\[0.4cm]

        % Date with icon-style bullet
        {\large\textcolor{accentcolor}{\textbullet\hspace{0.3em}2025-09-30\hspace{0.3em}\textbullet}} \\[0.3cm]

        \vspace{1cm}
    \end{center}
\end{tcolorbox}

\vspace{1.5cm}

% Content section with better spacing
\textbf{Baroque Music Analysis}
==========================

\subsubsection*{Key Characteristics}

\begin{itemize}
  \item \textbf{Lack of percussion}: Baroque music primarily features the cymbal as a percussive instrument.
  \item \textbf{Consistency throughout the piece}: Baroque music is known for its cohesive and unified sound, with minimal changes in mood or texture.
  \item \textbf{Polyphonic texture}: Characterized by multiple independent melodies sounding simultaneously, creating an eerie and complex sound.
\end{itemize}

\subsubsection*{Insights}

\begin{itemize}
  \item The emphasis on consistency and coherence in Baroque music may be due to the cultural and historical context of the time period. During the Baroque era, music was often performed for royalty and nobility, who valued elegance and refinement.
  \item The lack of percussion, apart from the cymbal, contributes to a sense of restraint and control in Baroque music.
\end{itemize}

\subsubsection*{Priorities}

\begin{itemize}
  \item \textbf{Understand the role of consistency}: Recognize how consistency is used as a musical device to create a unified sound and mood.
  \item \textbf{Analyze polyphonic texture}: Study how multiple independent melodies can come together to create a complex and interesting sound.
\end{itemize}

\subsubsection*{Actionable Items}

\begin{itemize}
  \item Listen to examples of Baroque music, paying close attention to the use of cymbal and polyphonic texture.
  \item Analyze specific pieces, such as Mozart's symphonies or Bach's cantatas, to see how consistency and coherence are used as musical devices.
  \item Research the cultural and historical context of the Baroque era to gain a deeper understanding of its musical characteristics.
\end{itemize}

\subsubsection*{Connections}

\begin{itemize}
  \item \textbf{Classical music}: The emphasis on consistency and coherence in Baroque music can be seen as a precursor to the Classical period's focus on balance and proportion.
  \item \textbf{Counterpoint}: Polyphonic texture is closely related to counterpoint, a musical technique used in various periods to create complex and interweaving melodies.
\end{itemize}

\vspace{2cm}

% Modern footer with branding
\begin{center}
\begin{tcolorbox}[
    colback=primarycolor!5,
    colframe=primarycolor!30,
    boxrule=0.5pt,
    arc=4pt,
    width=0.7\textwidth,
    left=10pt,
    right=10pt,
    top=6pt,
    bottom=6pt
]
    \centering
    \textcolor{primarycolor!80}{\small\textit{Generated by}} \textcolor{primarycolor}{\small\textbf{NoteTaker AI}} \\[-2pt]
    \textcolor{secondarycolor!60}{\tiny Your intelligent note-taking companion}
\end{tcolorbox}
\end{center}

\end{document}
