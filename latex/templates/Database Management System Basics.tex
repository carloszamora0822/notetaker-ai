\documentclass[11pt,letterpaper]{article}

% Modern packages
\usepackage[utf8]{inputenc}
\usepackage[T1]{fontenc}
\usepackage{lmodern}  % Better font
\usepackage{microtype}  % Better typography
\usepackage{xcolor}
\usepackage{geometry}
\usepackage{fancyhdr}
\usepackage{hyperref}
\usepackage{enumitem}  % Better lists
\usepackage{tcolorbox}  % Colorful boxes
\usepackage{graphicx}
\usepackage{amsmath,amssymb}  % Math support

% Page geometry with better margins
\geometry{
    margin=0.9in,
    top=1.1in,
    headheight=14pt
}

% Theme colors (will be replaced by generator)
\definecolor{primarycolor}{HTML}{28a745}
\definecolor{secondarycolor}{HTML}{66bb6a}
\definecolor{accentcolor}{HTML}{1E88E5}
\definecolor{backgroundcolor}{HTML}{F8F9FA}

% Modern header styling with gradient effect
\pagestyle{fancy}
\fancyhf{}
\fancyhead[L]{\textcolor{primarycolor}{\small\textbf{DATABASE MANAGEMENT SYSTEMS}}}
\fancyhead[R]{\textcolor{secondarycolor!80}{\small 2025-09-04}}
\fancyfoot[C]{\textcolor{primarycolor!60}{\small\thepage}}
\renewcommand{\headrulewidth}{0.5pt}
\renewcommand{\headrule}{\hbox to\headwidth{\color{primarycolor!30}\leaders\hrule height \headrulewidth\hfill}}

% Hyperref setup
\hypersetup{
    colorlinks=true,
    linkcolor=primarycolor,
    urlcolor=secondarycolor,
    citecolor=primarycolor,
    pdfborder={0 0 0}
}

% Modern section formatting with boxes
\usepackage{titlesec}
\titleformat{\section}
  {\Large\bfseries\color{primarycolor}}
  {}{0pt}{}
  [\vspace{-8pt}{\color{primarycolor}\titlerule[1.5pt]}]

\titleformat{\subsection}
  {\large\bfseries\color{secondarycolor}}
  {\textcolor{secondarycolor}{\textbullet}\hspace{0.3em}}{0pt}{}

\titleformat{\subsubsection}
  {\normalsize\bfseries\color{accentcolor}}
  {\textcolor{accentcolor}{\tiny\blacksquare}\hspace{0.3em}}{0pt}{}

% Title styling
\title{\textcolor{primarycolor}{\Huge\textbf{DATABASE MANAGEMENT SYSTEMS Notes}}}
\author{}
\date{\textcolor{secondarycolor}{2025-09-04}}

% Customize list styling
\setlist[itemize]{leftmargin=*, itemsep=3pt, topsep=6pt}
\setlist[enumerate]{leftmargin=*, itemsep=3pt, topsep=6pt}

% Custom box for highlights
\newtcolorbox{highlightbox}{%
    colback=secondarycolor!5,
    colframe=primarycolor,
    boxrule=0.5pt,
    arc=3pt,
    left=8pt,
    right=8pt,
    top=6pt,
    bottom=6pt
}

\begin{document}

% Modern title page with gradient background effect
\begin{tcolorbox}[
    enhanced,
    colback=backgroundcolor,
    colframe=primarycolor,
    boxrule=2pt,
    arc=8pt,
    width=\textwidth,
    left=0pt,
    right=0pt,
    top=0pt,
    bottom=0pt
]
    \begin{center}
        \vspace{1.2cm}
        % Main title with modern styling
        {\fontsize{32}{38}\selectfont\textcolor{primarycolor}{\textbf{DATABASE MANAGEMENT SYSTEMS}}} \\[0.8cm]

        % Decorative line
        {\color{secondarycolor}\rule{0.6\textwidth}{1.5pt}} \\[0.5cm]

        % Subtitle
        {\Large\textcolor{secondarycolor!80}{\textit{Lecture Notes}}} \\[0.4cm]

        % Date with icon-style bullet
        {\large\textcolor{accentcolor}{\textbullet\hspace{0.3em}2025-09-04\hspace{0.3em}\textbullet}} \\[0.3cm]

        \vspace{1cm}
    \end{center}
\end{tcolorbox}

\vspace{1.5cm}

% Content section with better spacing
\textbf{Database Management Systems (DBMS)}

\subsubsection*{Overview}
DBMS is a system that manages data by providing an interface between users and databases.

#### Why DBMS?
DBMS helps avoid:
\begin{itemize}
  \item Redundancy
  \item Inconsistency
  \item Security breaches
  \item Concurrency issues
  \item Data loss due to recovery
\end{itemize}

\subsubsection*{Data Models}
There are four main types of data models:

| \textbf{Data Model} | \textbf{Description} |
| --- | --- |
| Hierarchical | Organized as a tree-like structure (e.g., organizational charts) |
| Network | Organized as a graph-like structure (e.g., social networks) |
| Relational | Organized into tables with well-defined relationships between them (most common) |
| Object-Oriented | Organized around objects and their attributes, similar to programming languages |

#### Schema and Instance
\begin{itemize}
  \item \textbf{Schema}: A blueprint or conceptual representation of the database.
  \item \textbf{Instance}: A snapshot of the actual data in the database.
\end{itemize}

\subsubsection*{SQL}
SQL is a standard language for managing relational databases. It consists of:
\begin{itemize}
  \item \textbf{DDL (Data Definition Language)}: Used to create, modify, and delete database objects.
  \item \textbf{DML (Data Manipulation Language)}: Used to insert, update, delete, and select data.
  \item \textbf{DCL (Data Control Language)}: Used to grant or revoke permissions for users.
  \item \textbf{TCL (Transaction Control Language)}: Used to control transactions.
\end{itemize}

\subsubsection*{Normalization}
Normalization is the process of organizing data in a database to minimize data redundancy and improve data integrity. There are four normal forms:
\begin{enumerate}
  \item \textbf{1NF (Atomic)}: Each cell in a table contains a single value.
  \item \textbf{2NF (No Partial Dependency)}: A table is in 2NF if it has no partial dependencies.
  \item \textbf{3NF (No Transitive Dependencies)}: A table is in 3NF if it has no transitive dependencies.
  \item \textbf{BCNF (Boyce-Codd Normal Form)}: The most restrictive form, where each non-key attribute depends on the entire primary key.
\end{enumerate}

\subsubsection*{ACID Properties}
ACID properties ensure that database transactions are processed reliably:
\begin{itemize}
  \item \textbf{Atomicity}: Ensures that either all or none of the changes are made.
  \item \textbf{Consistency}: Ensures that the data remains in a consistent state.
  \item \textbf{Isolation}: Ensures that concurrent transactions do not interfere with each other.
  \item \textbf{Durability}: Ensures that once a transaction is committed, its effects are permanent.
\end{itemize}

\subsubsection*{Concurrency Control}
Concurrency control techniques ensure that multiple users can access and update data simultaneously:
\begin{itemize}
  \item \textbf{Locks}: Temporarily locks data to prevent conflicts.
  \item \textbf{Deadlocks}: Can occur when two or more transactions wait for each other to release resources.
  \item \textbf{Timestamps}: Assigns a timestamp to each transaction, allowing for conflict resolution.
  \item \textbf{Optimistic vs Pessimistic}: Optimistic approaches assume that concurrent updates will not conflict, while pessimistic approaches assume they will.
\end{itemize}

\subsubsection*{Entity-Relationship Model}
The ER model represents data as entities, attributes, and relationships:
\begin{itemize}
  \item \textbf{Entities}: Tables in the database.
  \item \textbf{Attributes}: Columns in a table.
  \item \textbf{Relationships}: Cardinality (number of instances) between tables.
\end{itemize}

\subsubsection*{Relational Algebra Operations}
Relational algebra operations are used to manipulate relational databases:
\begin{itemize}
  \item \textbf{Select σ}: Selects rows based on conditions.
  \item \textbf{Project π}: Projects attributes from a table.
  \item \textbf{Union ∪}: Combines two or more tables into one.
  \item \textbf{Intersection ∩}: Returns only the common elements between two tables.
  \item \textbf{Difference −}: Returns the difference between two tables.
\end{itemize}

\subsubsection*{Indexing}
Indexing speeds up query performance by providing quick access to data:
\begin{itemize}
  \item \textbf{B+ Trees}: A self-balancing search tree used for indexing.
  \item \textbf{Hash indexes}: Use a hash function to map keys to locations in memory.
  \item \textbf{Clustered vs Non-Clustered Indexes}: Clustered indexes store the actual data, while non-clustered indexes store only pointers.
\end{itemize}

\subsubsection*{Storage}
Storage involves managing data on disk:
\begin{itemize}
  \item \textbf{Files}: Contain data that is stored on disk.
  \item \textbf{Blocks}: A group of contiguous bytes on disk.
  \item \textbf{Pages}: A fixed-size block of data in memory.
\end{itemize}

\subsubsection*{Recovery}
Recovery techniques ensure data integrity after failures or crashes:
\begin{itemize}
  \item \textbf{Logging (Write-Ahead Log)}: Records changes to the database before they are written to disk.
  \item \textbf{Checkpoints}: Periodically saves the state of the database to disk.
  \item \textbf{Undo/Redo}: Allows for rolling back transactions in case of errors.
\end{itemize}

\subsubsection*{Distributed Database}
A distributed database is a collection of databases that are connected and accessed remotely:
\begin{itemize}
  \item \textbf{Fragmentation}: Occurs when data is scattered across multiple nodes, making it difficult to access.
  \item \textbf{Replication}: Copies data from one node to another to improve availability.
\end{itemize}

\subsubsection*{NoSQL Databases}
NoSQL databases are designed for large-scale data storage and retrieval:
\begin{itemize}
  \item \textbf{Key-Value Stores}: Store data as key-value pairs.
  \item \textbf{Document-Oriented Databases}: Store data in self-describing documents.
  \item \textbf{Column-Oriented Databases}: Store data in columns instead of rows.
  \item \textbf{Graph Databases}: Store data as nodes and edges.
\end{itemize}

\subsubsection*{Security}
Security measures protect against unauthorized access:
\begin{itemize}
  \item \textbf{Authentication}: Verifies user identity before granting access.
  \item \textbf{Authorization}: Determines what actions a user can perform on the system.
  \item \textbf{Access Control}: Regulates who has access to sensitive data or systems.
\end{itemize}

#### SQL Injection Vulnerability!
SQL injection occurs when an attacker injects malicious SQL code into a web application.

\vspace{2cm}

% Modern footer with branding
\begin{center}
\begin{tcolorbox}[
    colback=primarycolor!5,
    colframe=primarycolor!30,
    boxrule=0.5pt,
    arc=4pt,
    width=0.7\textwidth,
    left=10pt,
    right=10pt,
    top=6pt,
    bottom=6pt
]
    \centering
    \textcolor{primarycolor!80}{\small\textit{Generated by}} \textcolor{primarycolor}{\small\textbf{NoteTaker AI}} \\[-2pt]
    \textcolor{secondarycolor!60}{\tiny Your intelligent note-taking companion}
\end{tcolorbox}
\end{center}

\end{document}
